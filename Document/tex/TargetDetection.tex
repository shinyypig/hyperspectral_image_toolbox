\section{目标检测}
在普通的彩色图像中进行目标检测往往不仅需要目标的颜色信息,还需要目标的形态信息。这是因为普通的彩色图像只包含RGB三个波段,所包含的信息过少,目标与背景之间的区分度较低。比如真植被和假植被在彩色图像中都是绿色,很难区分。而多光谱或高光谱图像具有丰富的光谱维度信息,往往只需要获得目标光谱就足以进行目标检测。一般来说,针对一幅多(高)光谱图像,目标检测算法的目的是生成一副灰度图,其中目标区域的像素值与背景区域的像素值差异尽可能大。

\subsection{问题建模}
设有一幅光谱图像 \(\mathbf{H} \in \mathbb{R}^{m \times n \times l}\),即图像域大小为 \(m\times n\),具有 \(l\) 个波段。可以通过简单的reshape操作将其转换为一个 \(N \times L\) 的矩阵
\[
    \mathbf{X} = \begin{bmatrix}
        \bm{x}_{1}^{\mathrm{T}} \\
        \bm{x}_{2}^{\mathrm{T}} \\
        \vdots                  \\
        \bm{x}_{N}^{\mathrm{T}} \\
    \end{bmatrix}
\]
其中 \(N=m \times n\),\(L=l\),\(\bm{x}_{i} \in \mathbb{R}^{L \times 1}\) 为第 \(i\) 个像素的光谱。假设目标光谱为 \(\bm{d}\),那么目标检测的目标是找到一个映射函数 \(f(\cdot)\):
\[
    f(\bm{x}_{i}, \bm{d}) = \left\{
    \begin{aligned}
        1, & \quad \bm{x}_{i} = \bm{d}    \\
        0, & \quad \bm{x}_{i} \neq \bm{d}
    \end{aligned} \right.
\]

但在实际应用中,这样简单的模型是不够的。一是,图像中往往会存在噪声,与目标光谱 \(\bm{d}\) 完全相同的光谱在图像中可能就不存在,尽管确实存在目标。二是,在高光谱图像中往往会存在混像元,即一个像元中包含了多个地物的光谱。对于这些混像元,显然并不能简单地将其归类为目标像素或是非目标像素。

因此,有很多更加实用的模型提了出来,我们将会在下文中对一些常见的模型进行一一介绍。

\subsection{光谱角度映射(Spectral Angle Mapper)}
光谱角度映射(Spectral Angle Mapper,SAM)是一种基于光谱之间夹角的目标检测算法。其基本思想为:像元的光谱与目标光谱之间的夹角越小,说明该像元与目标光谱越相似,因此该像元越可能是目标像元。SAM 对光照强度不敏感,是一种非常简单实用的目标检测算法。SAM 的计算公式如下:
\[
    f(\bm{x}_{i}, \bm{d}) = \cos^{-1}\left(\frac{\bm{x}_{i}^{\mathrm{T}}\bm{d}}{\|\bm{x}_{i}\|\|\bm{d}\|}\right).
\]
需要注意的是,根据上式,\(f(\bm{x}_{i}, \bm{d})\) 越小,则说明 \(\bm{x}_{i}\) 与 \(\bm{d}\) 越相似。有的时候我们会省去 \(\cos^{-1}\),直接令
\[
    f(\bm{x}_{i}, \bm{d}) = \frac{\bm{x}_{i}^{\mathrm{T}}\bm{d}}{\|\bm{x}_{i}\|\|\bm{d}\|}.
\]
此时,\(f(\bm{x}_{i}, \bm{d})\) 越大,则说明 \(\bm{x}_{i}\) 与 \(\bm{d}\) 越相似。特别地,当 \(\bm{x}_{i} = \bm{d}\) 时,\(f(\bm{x}_{i}, \bm{d}) = 1\)。

有的时候,直接计算光谱夹角效果并不好。这是因为获得地物的光谱实际上是日光到达地面后的光谱再乘上地物的反射率,所以不同地物的光谱看起来非常接近(夹角很小)。因此,SAM 通常在校正后的反射率数据中使用。或者,我们可以先对高光谱图像进行白化,之后再对进行目标检测。

